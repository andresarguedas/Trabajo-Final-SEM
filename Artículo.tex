% Options for packages loaded elsewhere
\PassOptionsToPackage{unicode}{hyperref}
\PassOptionsToPackage{hyphens}{url}
\PassOptionsToPackage{dvipsnames,svgnames*,x11names*}{xcolor}
%
\documentclass[
]{article}
\usepackage{lmodern}
\usepackage{amssymb,amsmath}
\usepackage{ifxetex,ifluatex}
\ifnum 0\ifxetex 1\fi\ifluatex 1\fi=0 % if pdftex
  \usepackage[T1]{fontenc}
  \usepackage[utf8]{inputenc}
  \usepackage{textcomp} % provide euro and other symbols
\else % if luatex or xetex
  \usepackage{unicode-math}
  \defaultfontfeatures{Scale=MatchLowercase}
  \defaultfontfeatures[\rmfamily]{Ligatures=TeX,Scale=1}
\fi
% Use upquote if available, for straight quotes in verbatim environments
\IfFileExists{upquote.sty}{\usepackage{upquote}}{}
\IfFileExists{microtype.sty}{% use microtype if available
  \usepackage[]{microtype}
  \UseMicrotypeSet[protrusion]{basicmath} % disable protrusion for tt fonts
}{}
\makeatletter
\@ifundefined{KOMAClassName}{% if non-KOMA class
  \IfFileExists{parskip.sty}{%
    \usepackage{parskip}
  }{% else
    \setlength{\parindent}{0pt}
    \setlength{\parskip}{6pt plus 2pt minus 1pt}}
}{% if KOMA class
  \KOMAoptions{parskip=half}}
\makeatother
\usepackage{xcolor}
\IfFileExists{xurl.sty}{\usepackage{xurl}}{} % add URL line breaks if available
\IfFileExists{bookmark.sty}{\usepackage{bookmark}}{\usepackage{hyperref}}
\hypersetup{
  colorlinks=true,
  linkcolor=blue,
  filecolor=Maroon,
  citecolor=Blue,
  urlcolor=blue,
  pdfcreator={LaTeX via pandoc}}
\urlstyle{same} % disable monospaced font for URLs
\usepackage[margin=1in]{geometry}
\usepackage{color}
\usepackage{fancyvrb}
\newcommand{\VerbBar}{|}
\newcommand{\VERB}{\Verb[commandchars=\\\{\}]}
\DefineVerbatimEnvironment{Highlighting}{Verbatim}{commandchars=\\\{\}}
% Add ',fontsize=\small' for more characters per line
\usepackage{framed}
\definecolor{shadecolor}{RGB}{248,248,248}
\newenvironment{Shaded}{\begin{snugshade}}{\end{snugshade}}
\newcommand{\AlertTok}[1]{\textcolor[rgb]{0.94,0.16,0.16}{#1}}
\newcommand{\AnnotationTok}[1]{\textcolor[rgb]{0.56,0.35,0.01}{\textbf{\textit{#1}}}}
\newcommand{\AttributeTok}[1]{\textcolor[rgb]{0.77,0.63,0.00}{#1}}
\newcommand{\BaseNTok}[1]{\textcolor[rgb]{0.00,0.00,0.81}{#1}}
\newcommand{\BuiltInTok}[1]{#1}
\newcommand{\CharTok}[1]{\textcolor[rgb]{0.31,0.60,0.02}{#1}}
\newcommand{\CommentTok}[1]{\textcolor[rgb]{0.56,0.35,0.01}{\textit{#1}}}
\newcommand{\CommentVarTok}[1]{\textcolor[rgb]{0.56,0.35,0.01}{\textbf{\textit{#1}}}}
\newcommand{\ConstantTok}[1]{\textcolor[rgb]{0.00,0.00,0.00}{#1}}
\newcommand{\ControlFlowTok}[1]{\textcolor[rgb]{0.13,0.29,0.53}{\textbf{#1}}}
\newcommand{\DataTypeTok}[1]{\textcolor[rgb]{0.13,0.29,0.53}{#1}}
\newcommand{\DecValTok}[1]{\textcolor[rgb]{0.00,0.00,0.81}{#1}}
\newcommand{\DocumentationTok}[1]{\textcolor[rgb]{0.56,0.35,0.01}{\textbf{\textit{#1}}}}
\newcommand{\ErrorTok}[1]{\textcolor[rgb]{0.64,0.00,0.00}{\textbf{#1}}}
\newcommand{\ExtensionTok}[1]{#1}
\newcommand{\FloatTok}[1]{\textcolor[rgb]{0.00,0.00,0.81}{#1}}
\newcommand{\FunctionTok}[1]{\textcolor[rgb]{0.00,0.00,0.00}{#1}}
\newcommand{\ImportTok}[1]{#1}
\newcommand{\InformationTok}[1]{\textcolor[rgb]{0.56,0.35,0.01}{\textbf{\textit{#1}}}}
\newcommand{\KeywordTok}[1]{\textcolor[rgb]{0.13,0.29,0.53}{\textbf{#1}}}
\newcommand{\NormalTok}[1]{#1}
\newcommand{\OperatorTok}[1]{\textcolor[rgb]{0.81,0.36,0.00}{\textbf{#1}}}
\newcommand{\OtherTok}[1]{\textcolor[rgb]{0.56,0.35,0.01}{#1}}
\newcommand{\PreprocessorTok}[1]{\textcolor[rgb]{0.56,0.35,0.01}{\textit{#1}}}
\newcommand{\RegionMarkerTok}[1]{#1}
\newcommand{\SpecialCharTok}[1]{\textcolor[rgb]{0.00,0.00,0.00}{#1}}
\newcommand{\SpecialStringTok}[1]{\textcolor[rgb]{0.31,0.60,0.02}{#1}}
\newcommand{\StringTok}[1]{\textcolor[rgb]{0.31,0.60,0.02}{#1}}
\newcommand{\VariableTok}[1]{\textcolor[rgb]{0.00,0.00,0.00}{#1}}
\newcommand{\VerbatimStringTok}[1]{\textcolor[rgb]{0.31,0.60,0.02}{#1}}
\newcommand{\WarningTok}[1]{\textcolor[rgb]{0.56,0.35,0.01}{\textbf{\textit{#1}}}}
\usepackage{graphicx}
\makeatletter
\def\maxwidth{\ifdim\Gin@nat@width>\linewidth\linewidth\else\Gin@nat@width\fi}
\def\maxheight{\ifdim\Gin@nat@height>\textheight\textheight\else\Gin@nat@height\fi}
\makeatother
% Scale images if necessary, so that they will not overflow the page
% margins by default, and it is still possible to overwrite the defaults
% using explicit options in \includegraphics[width, height, ...]{}
\setkeys{Gin}{width=\maxwidth,height=\maxheight,keepaspectratio}
% Set default figure placement to htbp
\makeatletter
\def\fps@figure{htbp}
\makeatother
\setlength{\emergencystretch}{3em} % prevent overfull lines
\providecommand{\tightlist}{%
  \setlength{\itemsep}{0pt}\setlength{\parskip}{0pt}}
\setcounter{secnumdepth}{-\maxdimen} % remove section numbering


%\usepackage{fancyhdr}
%\pagestyle{fancy}
%\rhead{\includegraphics[width = 1\textwidth]{marca.jpg}}


\usepackage{geometry}
\geometry{a4paper, left=10mm, right=10mm, bottom=15mm}
\usepackage{setspace}
\doublespacing
\usepackage[spanish]{babel}
\usepackage{color}
\usepackage{xcolor}
\usepackage{framed}
\colorlet{shadecolor}{gray!20}
\setcounter{secnumdepth}{0}
\usepackage{sectsty}


\chapternumberfont{\Large}
\chaptertitlefont{\Large}
\setcounter{tocdepth}{5}
\setcounter{secnumdepth}{5}
\setlength{\footskip}{20pt}%Esto sube el número de página
\usepackage{graphics}
\usepackage{setspace} %paquete para el doble espaciado
\doublespacing %inicia el doble espaciado
 %Esto quita el punto final en la numeracion de cada seccion
\usepackage{tocloft}

\usepackage{titlesec}
\titleformat{\section}
{\Large\bfseries}{\thesection}{0.5em}{}
\titleformat{\subsection}
{\large\bfseries}{\thesubsection}{0.5em}{}
\titleformat{\subsubsection}
{\normalsize\bfseries}{\thesubsubsection}{0.5em}{}
\titleformat{\paragraph}
{\normalsize\bfseries}{\theparagraph}{0.5em}{}
\renewcommand\cftsecaftersnum{}
\renewcommand\thesection{\arabic{section}}
\renewcommand\thesubsection{\thesection.\arabic{subsection}}
\usepackage{caption}
\usepackage{fancyhdr}
\pagestyle{fancy}
\fancyhf{}
\fancyhead[R]{\rightmark}
\fancyfoot[R]{\thepage}
%\fancyfoot[C]{Teléfono  2511-1400    /    posgrado@sep.ucr.ac.cr  /   www.sep.ucr.ac.cr}
\setlength{\headheight}{21.9pt}
\renewcommand\sectionmark[1]{%
\markright{\thesection\ #1}}
%\renewcommand{\footrulewidth}{0.4pt}


%\renewcommand{\footnoterule}{%
%  \kern -1pt
%  \hrule width \textwidth height 1pt
%  \kern 4pt
%}


%MARCA DE AGUA
%\usepackage{graphicx}
% \usepackage{fancyhdr}
%  \pagestyle{fancy}
%  \setlength\headheight{28pt}
%   \fancyhead[L]{\includegraphics[width=16cm]{marca.jpg}}
%   \fancyfoot[LE,RO]{}

\usepackage{booktabs}
\usepackage{longtable}
\usepackage{array}
\usepackage{multirow}
\usepackage{wrapfig}
%\usepackage{float}
\usepackage{colortbl}
\usepackage{pdflscape}
\usepackage{tabu}
\usepackage{threeparttable}
\usepackage{threeparttablex}
\usepackage[normalem]{ulem}
\usepackage{makecell}
\usepackage{xcolor}

\usepackage{tocloft}
\renewcommand{\cftsecleader}{\cftdotfill{\cftdotsep}}

%\renewcommand{\familydefault}{\sfdefault} %Para cambiar la fuente


%Para referenciar chunks
\usepackage{caption}
\usepackage{floatrow}
\floatsetup[figure]{capposition=top}
\floatsetup[table]{capposition=top}
\floatplacement{figure}{H}
\floatplacement{table}{H}

\DeclareNewFloatType{chunk}{placement=H, fileext=chk, name=}
\captionsetup{options=chunk}
\renewcommand{\thechunk}{Código~\arabic{chunk}}
\makeatletter
\@addtoreset{chunk}{section}
\makeatother

%Paquetes adicionales para ecuaciones, símbolos y figuras
\usepackage{amssymb}
\usepackage{tikz}
\usetikzlibrary{babel,positioning,shapes.multipart,calc,arrows.meta}
\usepackage{booktabs}
\usepackage{longtable}
\usepackage{array}
\usepackage{multirow}
\usepackage{wrapfig}
\usepackage{float}
\usepackage{colortbl}
\usepackage{pdflscape}
\usepackage{tabu}
\usepackage{threeparttable}
\usepackage{threeparttablex}
\usepackage[normalem]{ulem}
\usepackage{makecell}
\usepackage{xcolor}
\newlength{\cslhangindent}
\setlength{\cslhangindent}{1.5em}
\newenvironment{cslreferences}%
  {\setlength{\parindent}{0pt}%
  \everypar{\setlength{\hangindent}{\cslhangindent}}\ignorespaces}%
  {\par}

\title{UNIVERSIDAD DE COSTA RICA\\
SISTEMA DE ESTUDIOS DE POSGRADO\\
ESCUELA DE ESTADÍSTICA\\
~\\
~\\}
\usepackage{etoolbox}
\makeatletter
\providecommand{\subtitle}[1]{% add subtitle to \maketitle
  \apptocmd{\@title}{\par {\large #1 \par}}{}{}
}
\makeatother
\subtitle{EFECTOS DE LA KURTOSIS EN LA ESTIMACIÓN DE MODELOS DE
ECUACIONES ESTRUCTURALES BAJO DISTINTOS TAMAÑOS DE MUESTRA\\
~\\
~\\
~\\}
\author{\hfill\break
\hfill\break
\hfill\break
\hfill\break
\hfill\break
CÉSAR ANDRÉS GAMBOA SANABRIA B12672\\
ANDRÉS ESTEBAN ARGUEDAS LEIVA B40535\\
~\\
~\\
~\\
~\\
Ciudad Universitaria Rodrigo Facio, Costa Rica\\
~\\
~\\}
\date{2020}

\begin{document}
\maketitle

\pagenumbering{gobble}
\cleardoublepage

\newpage

\section*{RESUMEN}
\pagenumbering{gobble}

El Resumen

\textbf{\emph{Palabras clave}}:SEM, simulación, kurtosis, lavaan
\cleardoublepage

\section*{ABSTRACT}
\pagenumbering{gobble}

El resumen en inglés

\textbf{\emph{Palabras clave}}:SEM, simulation, kurtosis, lavaan
\cleardoublepage

\newpage

\pagenumbering{gobble}
\tableofcontents
\cleardoublepage
\pagenumbering{arabic}

\newpage

\section{INTRODUCCIÓN}

\subsection{Antecedentes}

Los Modelos de Ecuaciones Estructurales (en adelante SEM, por sus siglas
en inglés) representan un compendio de métodos estadísticos que buscan
estimar y examinar las relaciones causales existentes entre varias
mediciones fácilmente observables con conceptos más abstractos,
denominados constructos, que no pueden ser medidos ni analizados de
manera directa. Los SEM trabajan de una manera similar a los modelos de
regresión más clásicos, pero representan una mejora pues analizan las
relaciones causales lineales entre las variables involucradas al mismo
tiempo que los errores de medición (Beran \& Violato,
\protect\hyperlink{ref-Beran2010StructuralEM}{2010}). Para medir estas
relaciones causales, los SEM cuentan con dos grandes componentes: 1) El
modelo estructural, cuya función es cuantificar las relaciones causales
presentes entre cada uno de los constructos planteados desde la teoría;
y 2) un modelo de medición, cuyo objetivo último es brindar una
descripción acerca de cuáles son los indicadores que sirven para medir
los constructos en cuestión (Kaplan,
\protect\hyperlink{ref-kaplan}{2008}).

Los SEM están presentes en multitud de campos de investigación como la
psicología, la sociología, las políticas públicas y ciencias
relacionadas a la familia (Tarka,
\protect\hyperlink{ref-tarka}{2018}\protect\hyperlink{ref-tarka}{a}),
además, trabajos como el de Thomas Golob (Golob,
\protect\hyperlink{ref-golob}{2011}) muestran la aplicación de los SEM
en fenómenos económicos, o bien en investigación de mercados como
sugieren los trabajos de Bagozzi (Horton,
\protect\hyperlink{ref-bagozzi}{1980}) y Chin (Chin, Peterson, \& Brown,
\protect\hyperlink{ref-chin}{2008}). Según Beran y Violato (2010), la
cantidad de referencias a SEM en 1994 fueron de 164, aumentaron a 343 en
el 2000 y llegaron a 742 en el 2009, lo cual es una señal de que muchos
investigadores alrededor del mundo están mostrando cada vez más interés
en este tipo de estudios, pues representan una potente herramienta para
la investigación partiendo de la teoría sustantiva que poseen los
diversos estudios.

Uno de los principales campos de aplicación de los SEM son las ciencias
sociales, pues se busca explicar y/o predecir con un grado de validez el
comportamiento específico de una o varias personas en grupo. Teniendo
siempre en consideración (aunque de forma limitada) las condiciones que
afectan a cada individuo involucrado en el estudio, así como las
características propias de su entorno, los grupos de investigación
pueden definir factores, las relaciones latentes y de causalidad entre
ellos que se encuentran implícitas en el comportamiento humano. Este
tipo de investigaciones permite entender los fenómenos no solo de forma
descriptiva, sino que es posible también determinar relaciones de
causalidad (Tarka,
\protect\hyperlink{ref-Tarka}{2018}\protect\hyperlink{ref-Tarka}{b}).

Las variables indicadoras que sirven para construir las relaciones
implícitas en cuestión, estas relaciones implícitas son llamadas
comúnmente llamadas constructos, pueden llegar a comportarse de manera
muy diversa. Las ciencias sociales, al trabajar con seres humanos, es
común trabajar con variables cuyo comportamiento es particularmente
irregular, presentando valores muy distintos entre los sujetos de
estudio, generando de esta manera que los indicadores de manera
multivariada no sigan una distribución normal, lo cual representa un
supuesto fundamental al trabajar con SEM (Sura-Fonseca,
\protect\hyperlink{ref-sura}{2020}),esta condición puede afectar
negativamente la estimación del modelo y sus estadísticos de bondad de
ajuste, llevando a pérdidas en la potencia (Foss, Jöreskog, \& Olsson,
\protect\hyperlink{ref-foss}{2011}) o al caso de descartar modelos que
podrían ser adecuados solo por presentar un mal ajuste (Andreassen,
Lorentzen, \& Olsson, \protect\hyperlink{ref-andreassen}{2006}). El no
cumplimiento de este supuesto puede deberse, entre otras cosas, a
medidas particularmente altas o bajas de una medida estadística en
específico: La kurtosis.

\subsection{El problema}

Si al trabajar con un SEM no se cumple el supuesto de normalidad
multivariada y además el modelo se estima vía máxima verosimilitud, que
al día de hoy se mantiene como el método de estimación más extendido,
podría cometerse el error de sobreestimar el estadístico chi-cuadrado,
el cual sirve de referencia para conocer la magnitud de la diferencia
entre la matriz de covariancias estimadas por el modelo con la obtenida
en la muestra. Lo anterior suele llevar al rechazar modelos que en
realidad resumen bien la realidad para dar una mejor explicación del por
qué sucede un fenómeno, y además a la subestimación de los errores
asociados a los parámetros, lo cual genera interpretaciones inadecuadas
en lo referente a la significancia estadística de las relaciones
planteadas por el modelo teórico.

Es posible toparse con conjuntos de datos que, en su conjunto, no
presenten una distribución normal multivariada debido a la muy alta o
muy baja concentración de datos alrededor de la zona central de su
distribución. Este comportamiento se mide mediante un estadístico
llamado kurtosis, que describe qué tan aplanada o empinada es la
distribución, dependiendo de este estadístico, es posible saber si los
datos atentan contra la presencia de una distribución normal. Trabajos
como el de Sura-Fonseca (Sura-Fonseca,
\protect\hyperlink{ref-sura}{2020}) o el de Tor (Andreassen et~al.,
\protect\hyperlink{ref-andreassen}{2006}) han abierto brechas de
investigación para esta problemática.

Considerar distintos niveles de kurtosis permite conocer el impacto que
esta medida tiene sobre las estimaciones de un SEM dependiendo del
tamaño de muestra utilizado (Muthen \& Kaplan,
\protect\hyperlink{ref-muthen}{1992}).

\subsection{Objetivos del estudio}

La presente investigación busca estudiar el efecto que tienen distintos
niveles de kurtosis en varios tamaños de muestra sobre las estimaciones
de un SEM. Para ello, se ha tomado tomado como base un estudio de la
Universidad de California (Gao, Mokhtarian, \& Johnston,
\protect\hyperlink{ref-gao}{2008}), por ser uno de los trabajos más
recientes en cuanto a planteamiento de tamaños de muestra y kurtosis
para la simulación de datos multivariados. Se plantean los siguientes
objetivos:

\subsubsection{Objetivo general}

Comparar mediante un estudio de simulación los efectos en las
estimaciones de cargas factoriales y medidas de ajuste de modelos de
ecuaciones estructurales estimados mediante máxima verosimilitud en
presencia de variables observadas con niveles de kurtosis de 0, 0.62,
6.65, 21.41 y 13.92 en tamaños de muestra de 50, 100, 120, 200 y 300.

\subsubsection{Objetivos específicos}

\textbf{1)} Definir como modelo poblacional el obtenido por Sura-Fonseca
(2020) con dos variables exógenas y una endógena con tres variables
indicadoras cada uno como modelo de referencia teórico cuyas cargas
factoriales se utilizarán para la generación de los datos simulados.

\textbf{2)} Medir el posible sesgo causado en la estimación de los
modelos mediante el estadístico chi-cuadrado del modelo y la raíz del
cuadrado medio de error de aproximación (RMSEA), la raíz de residuos de
cuadrado medio estandarizado (SRMR) y el índice de bondad de ajuste
(GFI).

\textbf{3)} Comparar los valores poblacionales de las cargas factoriales
con los obtenidos en las simulaciones.

\textbf{4)} Publicar en una revista científica con revisión por pares el
manuscrito final, en forma de un artículo científico.

\subsection{Metodología de la investigación}

De esta manera, el presente estudio consiste en en simular datos no
normales multivariados con diferentes tamaños de muestra y kurtosis para
la estimación de SEM tomando como punto modelo de referencia el obtenido
por Sura-Fonseca para las habilidades cuantitativas, el cual consiste en
dos variables exógenas y una endógena. Se realizaron 2000 conjuntos de
datos para cada escenario de simulación y se comparan las estimaciones
de tanto de las cargas factoriales como de varios estadísticos de bondad
de ajuste que serán descritos más adelante.

\newpage

\section{METODOLOGÍA}

\subsection{Generación de datos con kurtosis}

Los datos fueron simulados mediante la función \texttt{simulateData()}
del paquete \texttt{lavaan} (Rosseel,
\protect\hyperlink{ref-lavaan}{2012}), el cual utiliza el método
propuesto por Vale y Maurelli (Vale \& Maurelli,
\protect\hyperlink{ref-Vale1983}{1983}) para la simulación de datos no
normales multivariados. Este método, comúnmente conocido como VM, se
basa en el método propuesto por Fleishman (Fleishman,
\protect\hyperlink{ref-Fleishman1978}{1978}), el cual, con base en una
variable aleatoria distribuida como una normal estándar, permite simular
una variable con un promedio, variancia, asimetría y kurtosis dada. El
método VM permite especificar, adicionalmente, correlaciones entre las
variables a estimar. Para utilizar el método de Fleishman, para generar
una cierta variable aleatoria \(Y\), se utiliza la siguiente ecuación:

\begin{equation} \label{eq:defY}
  Y = a + bX + cX^2 + d X^3
\end{equation}

donde \(X \sim \mathcal{N} (0,1)\). Es decir, se puede generar una
variable no normal \(Y\), con sus primeros cuatro momentos iguales a
valores especificados, con base en los valores \(a\), \(b\), \(c\) y
\(d\) de la ecuación \ref{eq:defY}, con base en una variable normal
estándar \(X\) hasta su tercer potencia. Luego, para poder obtener los
valores de \(a\), \(b\), \(c\) y \(d\), se necesitan resolver las
siguientes ecuaciones de forma simultánea:

\begin{align*}
  b^2 + 6bd + 2c^2 + 15d^2 -1 & = 0 \\
  2c (b^2 + 24bd + 105d^2 + 2) - \gamma_1 & = 0 \\
  24 \left(bd + c^2 (1 + b^2 + 28bd) + d^2 (12 + 48bd + 141c^2 + 225d^2) \right) - \gamma_2 & = 0
\end{align*}

donde \(\gamma_1\) es la asimetría deseada y \(\gamma_2\) es la kurtosis
deseada, además se define \(a = -c\). Con base en las constantes
calculadas \(a\), \(b\), \(c\) y \(d\), además de una variable normal
estándar, se puede simular variables no normales. Para poder generalizar
el método de Fleishman a variables aleatorias multivariantes, Vale y
Maurelli proponen una generalización. Esta se basa, para el caso
bivariado, en la generación de dos variables aleatorias independientes,
\(X_1, X_2 \sim \mathcal{N} (0,1)\), para la cuales se obtienen las
constantes \(a\), \(b\), \(c\) y \(d\), para cada una de dichas
variables, como se describe en el método de Fleishman, obteniendo así el
vector \(w^\prime_1 = (a_1, b_1, c_1, d_1)\), para el caso de \(X_1\), y
el vector \(w^\prime_2 = (a_2, b_2, c_2, d_2)\), para el caso de
\(X_2\). Además, se definen los vectores
\(x_1^\prime = (1, X_1, X_1^2, X_1^3)\) y
\(x_2^\prime = (1, X_2, X_2^2, X_2^3)\). Por lo tanto, se pueden crear
variables no normales, \(Y_1\) y \(Y_2\), como:

\begin{align*}
  Y_1 & = w_1^\prime x_1 \\
  Y_2 & = w_2^\prime x_2
\end{align*}

donde se puede verificar que:

\begin{align*}
  r_{Y_1, Y_2} = & \rho_{X_1, X_2} (b_1 b_2 + 3b_1 d_2 + 3d_1 b_2 + 9 d_1 d_2) \\
  & + \rho_{X_1, X_2}^2 (2 c_1 c_2) + \rho_{X_1, X_2}^3 (6 d_1 d_2)
\end{align*}

Y resolviendo esta ecuación en términos de \(\rho_{X_1, X_2}\) se puede
obtener una matriz de correlaciones para generar datos normales
multivariados, que pueden ser transformados en variables no normales
mediante el método de Fleishman.

\subsection{Modelo a estimar}

El modelo teórico utilizada para realizar las simulaciones es el
presentado por (Sura-Fonseca, \protect\hyperlink{ref-sura}{2020}),
basado en datos de 155 estudiantes de la Universidad de Costa Rica,
obtenidos de la Prueba de Habilidades Cuantitativas (PHC) del Instituto
de Investigaciones Psicológicos (IIP) de dicha universidad y de un
cuestionario autoadministrado aplicado a estos estudiantes. El modelo
estimado está compuesto por dos variables exógenas (capital y
habilidades cuantitativas) y una variable endógena (habilidades
visoespaciales). Con respecto a estas variables: el capital se refiere
al acceso y tenencia de ciertos bienes en los hogares de los
estudiantes; las habilidades cuantitativas se refieren a la puntuación
de los estudiantes en la prueba mencionada anteriormente; y las
habilidades visoespaciales se refieren a la capacidad de los estudiantes
para poder trabajar con objetos tridimensionales abstractos y poder
manipularlos en la imaginación. Para cada una de estas variables
latentes, se utilizó el método de parcelas para obtener tres variables
indicadoras para cada uno de los constructos. Los resultados de la
estimación de dicho modelo, presentados por (Sura-Fonseca,
\protect\hyperlink{ref-sura}{2020}), se presentan en la Figura
\ref{fig:mod_teorico}.

\begin{figure}[!h]
\centering
\caption{Modelo estimado sobre las habilidades cuantitativas}
\label{fig:mod_teorico}
\begin{tikzpicture}
        [
    basic/.style={draw, text centered},
    circ/.style={basic, circle, minimum size=2em, inner sep=1.5pt},
    rect/.style={basic, text height=1em, text depth=.5em, minimum width=1.5em},
    >={Stealth[]}
    ]

    % Variables latentes
    \node [circ] (ve) {VE};
    \node [circ, above left=2cm and 2cm of ve] (ca) {CA};
    \node [circ, below left=2cm and 2cm of ve] (phc) {PHC};
    \draw [->] (ca) -- node[fill=white] {-0.01} (ve);
    \draw [->] (phc) -- node[fill=white] {0.41} (ve);
    % Variancia de VE
    \node [left=of ve] (delta) {0.83};
    \draw [->] (delta) -- (ve);
    % Variables indicadoras de VE
    \node [rect, above right=1.5cm and 3cm of ve.center] (pve1) {PVE1};
    \node [rect, right=3cm of ve.center] (pve2) {PVE2};
    \node [rect, below right=1.5cm and 3cm of ve.center] (pve3) {PVE3};
    \draw [->] (ve) -- node[fill=white] {0.62} (pve1.west);
    \draw [->] (ve) -- node[fill=white] {0.74} (pve2.west);
    \draw [->] (ve) -- node[fill=white] {0.72} (pve3.west);
    % Variancia de variables indicadoras de VE
    \node [right=of pve1] (epve1) {0.62};
    \node [right=of pve2] (epve2) {0.45};
    \node [right=of pve3] (epve3) {0.48};
    \draw [->] (epve1) -- (pve1);
    \draw [->] (epve2) -- (pve2);
    \draw [->] (epve3) -- (pve3);
    % Variables indicadoras de CA
    \node [rect, above left=1.5cm and 3cm of ca.center] (pc1) {PC1};
    \node [rect, left=3cm of ca.center] (pc2) {PC2};
    \node [rect, below left=1.5cm and 3cm of ca.center] (pc3) {PC3};
    \draw [->] (ca) -- node[fill=white] {0.88} (pc1.east);
    \draw [->] (ca) -- node[fill=white] {0.77} (pc2.east);
    \draw [->] (ca) -- node[fill=white] {0.73} (pc3.east);
    % Variancia de variables indicadoras de CA
    \node [left=of pc1] (epc1) {0.22};
    \node [left=of pc2] (epc2) {0.41};
    \node [left=of pc3] (epc3) {0.47};
    \draw [->] (epc1) -- (pc1);
    \draw [->] (epc2) -- (pc2);
    \draw [->] (epc3) -- (pc3);
    % Variables indicadoras de PHC
    \node [rect, above left=1.5cm and 3cm of phc.center] (phc1) {PHC1};
    \node [rect, left=3cm of phc.center] (phc2) {PHC2};
    \node [rect, below left=1.5cm and 3cm of phc.center] (phc3) {PHC3};
    \draw [->] (phc) -- node[fill=white] {0.85} (phc1.east);
    \draw [->] (phc) -- node[fill=white] {0.80} (phc2.east);
    \draw [->] (phc) -- node[fill=white] {0.82} (phc3.east);
    % Variancia de variables indicadoras de CA
    \node [left=of phc1] (ephc1) {0.28};
    \node [left=of phc2] (ephc2) {0.37};
    \node [left=of phc3] (ephc3) {0.33};
    \draw [->] (ephc1) -- (phc1);
    \draw [->] (ephc2) -- (phc2);
    \draw [->] (ephc3) -- (phc3);
\end{tikzpicture}
\end{figure}

\subsection{Simulación y estimación}

La simulación de los datos, junto con la estimación de los modelos, se
realizó mediante el paquete \texttt{lavaan} (Rosseel,
\protect\hyperlink{ref-lavaan}{2012}) usando el software R (R Core Team,
\protect\hyperlink{ref-R}{2020}) mediante la interfaz gráfica de RStudio
(RStudio Team, \protect\hyperlink{ref-RStudio}{2015}). Para el manejo de
bases de datos y demás visualizaciones fueron utilizados los paquetes
\texttt{ggplot2}(Wickham, \protect\hyperlink{ref-ggplot2}{2016}),
\texttt{tidyr} (Wickham \& Henry, \protect\hyperlink{ref-tidyr}{2020}),
\texttt{dplyr} (Wickham et~al., \protect\hyperlink{ref-dplyr}{2020}),
\texttt{ggpubr} (Kassambara, \protect\hyperlink{ref-ggpubr}{2020}),
\texttt{PerformanceAnalytics} (Peterson \& Carl,
\protect\hyperlink{ref-PerformanceAnalytics}{2020}) y
\texttt{kableExtra} (Zhu, \protect\hyperlink{ref-kableExtra}{2019}).

Para poder realizar la simulación deben seguirse varios pasos. Lo
primero es definir el modelo teórico poblacional que van a seguir los
datos simulados, como se describió en la sección anterior este modelo
cuenta con dos variables exógenas y una endógena, cada una con tres
variables indicadoras. Los datos se generan mediante la función
\texttt{simulateData()} la cuál requiere especificar varios argumentos,
uno de ellos es el modelo poblacional, cuya sintaxis puede encontrarse
en el \ref{modelo}. Los otros dos argumentos a especificar son el tamaño
de muestra deseado y el nivel de kurtosis de interés, la definición de
estos escenarios se obtiene mediante el \ref{escenarios} y se muestran
en el cuadro \ref{tab:escenarios}:

\begin{table}[!h]

\caption{\label{tab:unnamed-chunk-6}\label{tab:escenarios}Escenarios de simulación}
\centering
\resizebox{\linewidth}{!}{
\fontsize{9}{11}\selectfont
\begin{tabu} to \linewidth {>{\raggedleft}X>{\raggedleft}X>{\raggedleft}X>{\raggedleft}X>{\raggedleft}X>{\raggedleft}X>{\raggedleft}X>{\raggedleft}X>{\raggedleft}X>{\raggedleft}X}
\toprule
kurtosis & n & kurtosis & n & kurtosis & n & kurtosis & n & kurtosis & n\\
\midrule
\rowcolor{gray!6}  0.00 & 50 & 0.00 & 100 & 0.00 & 200 & 0.00 & 400 & 0.00 & 800\\
0.62 & 50 & 0.62 & 100 & 0.62 & 200 & 0.62 & 400 & 0.62 & 800\\
\rowcolor{gray!6}  6.65 & 50 & 6.65 & 100 & 6.65 & 200 & 6.65 & 400 & 6.65 & 800\\
21.41 & 50 & 21.41 & 100 & 21.41 & 200 & 21.41 & 400 & 21.41 & 800\\
\rowcolor{gray!6}  13.92 & 50 & 13.92 & 100 & 13.92 & 200 & 13.92 & 400 & 13.92 & 800\\
\bottomrule
\multicolumn{10}{l}{\textit{Fuente:} Elaboración propia a partir del estudio de la Universidad de California (2008)}\\
\end{tabu}}
\end{table}

Con estos escenarios definidos, se generaron entonces, para cada
combinación de tamaño de muestra y kurtosis un total de 2000 conjuntos
de datos para cada uno. La sintaxis necesaria para generar esto se
muestra en el \ref{simulacion}. Una vez que se obtuvieron estos
conjuntos de datos, el siguiente paso es realizar la estimación de los
SEM con cada uno de ellos; para ello es necesario definir un modelo de
una forma similar a como se indicó en el \ref{modelo}, pero esta vez sin
los valores de las cargas factoriales, pues se busca conocer las
estimaciones a partir de los datos generados, este proceso se muestra en
el \ref{modelos_sem}.

\subsection{Medidas de bondad de ajuste}

Las medidas de bondad de ajuste utilizadas para comparar el ajuste de
los modelos, para cada uno de los escenarios de simulación son: el
estadístico chi-cuadrado, el RMSEA, el SRMR y el CFI.

\subsubsection{Estadístico chi-cuadrado}

El estadístico de chi-cuadrado busca cuantificar la diferencia que se
presenta entre la matriz de covariancias de una muestra con la matriz de
covariancias estimadas mediante un cierto modelo. Según (Hu \& Bentler,
\protect\hyperlink{ref-Hu1999}{1999}), su fórmula de cálculo viene dada
por: \[
  \chi^2 = (N-1) F_{min}
\] donde \(N\) es el tamaño de la muestra y \(F_{min}\) es el mínimo
obtenido mediante la función de ajuste, la cual, normalmente, se asume
que es la distribución normal multivariada, utilizando el método de
máxima verosimilitud. Este estadístico tiene una distribución
chi-cuadrado con grados de libertad igual a la cantidad de piezas de
información única en la matriz de covariancias menos la cantidad de
parámetros a estimar del moedlo, bajo el supuesto de normalidad y, si
este supuesto no se cumple, la distribución asintótica sigue siendo una
chi-cuadrado con esos mismos grados de libertad. El estadístico
chi-cuadrado es muy utilizado en los modelos de ecuaciones estructurales
y da origen a la gran mayoría de las demás medidas de ajuste utilizadas
en dichos modelos, aunque puede presentar algunos problemas ya que
depende del tamaño de la muestra, por lo que, con muestras grandes
tiende a ser significativo, mientras que con muestras pequeños tiende a
no ser significativo (Kenny, \protect\hyperlink{ref-Kenny2015}{2015}).

\subsubsection{RMSEA}

El Error Cuadrático Medio de Aproximación (RMSEA por sus siglas en
inglés) es una de las medidas de ajuste más conocidas y utilizadas en
los modelos de ecuaciones estructurales. Su fórmula, según (Hu \&
Bentler, \protect\hyperlink{ref-Hu1999}{1999}), viene dada por: \[
  RMSEA = \sqrt{\max\left\{\frac{\chi^2 - gl}{gl (N-1)} , 0 \right\}}
\] donde \(\chi^2\) es el valor de la chi-cuadrado, \(gl\) son los
grados de libertad, y \(N\) es el tamaño de la muestra. Por lo general,
se considera un valor del RMSEA menor a 0.05 como un indicador de un
buen ajuste, mientras que un valor mayor a 0.1 representa un mal ajuste
del modelo (Kenny, \protect\hyperlink{ref-Kenny2015}{2015}).

\subsubsection{SRMR}

La Raíz Estandarizada del Error Cuadrático Medio (SRMR por sus siglas en
inglés) es una medida de ajuste en la cual se comparan las diferencias
entre las covariancias estimadas y las de la muestra. La fórmula de
cálculo, con base en (Hu \& Bentler,
\protect\hyperlink{ref-Hu1999}{1999}) es: \[
  SRMR = \sqrt{\left(2 \sum\limits_{i=1}^p \sum\limits_{j=1}^i \left((s_{ij} - \hat\sigma_{ij}) / (s_{ii} s_{jj}) \right)^2 \right)^2 / p(p+1)}
\] donde \(p\) es el número de variables observadas, \(s_{ij}\) son las
covariancias observadas y \(\hat\sigma_{ij}\) son las covariancias
estimadas de las variables \(i\) y \(j\). Dado que se están comparando
las covariancias observadas y las estimadas, un valor de 0 indica un
ajuste perfecto del modelo, pero, por lo general, se considera un valor
menor a 0.08 como un indicador de un buen ajuste (Kenny,
\protect\hyperlink{ref-Kenny2015}{2015}).

\subsubsection{CFI}

El Índice de Ajuste Comparativo (CFI por sus siglas en inglés) es una
medida de ajuste que compara el valor de chi-cuadrado del modelo
estimado con el valor de chi-cuadrado del modelo nulo, agregando una
penalización por la cantidad de parámetros estimados. La fórmula de
cálculo presenta por (Hu \& Bentler,
\protect\hyperlink{ref-Hu1999}{1999}) es la siguiente: \[
  CFI = 1 - \left( \frac{\max \left\{(\chi^2_T - gl_T), 0  \right\}}{\max \left\{(\chi^2_T - gl_T), (\chi^2_N - gl_N), 0 \right\}} \right)
\] donde \(\chi^2_T\) y \(\chi^2_N\) son los valores del estadístico
chi-cuadrado para el modelo estimado y el nulo, respectivamente, y
\(gl_T\) y \(gl_N\) son los grados de libertad de los modelos estimado y
nulo, respectivamente. Esta medida de ajuste puede tomar un valor entre
0 y 1 y se considera que el modelo tiene un buen ajuste cuando es mayor
a 0.95, un buen ajuste cuando el valor está entre 0.9 y 0.95 y un mal
ajuste cuando es menor que 0.9 (Kenny,
\protect\hyperlink{ref-Kenny2015}{2015}).

\newpage

\section{RESULTADOS}

\subsection{Introducción}

\newpage

\section{CONCLUSIONES Y RECOMENDACIONES}

\subsection{Introducción}

\subsection{Conclusiones}

\subsection{Recomendaciones}

\newpage

\section{ANEXOS}

\subsection{Modelo poblacional para la función simulateData()}

\captionof{chunk}{Modelo poblacional para simular datos}\label{modelo}

\begin{Shaded}
\begin{Highlighting}[]
\NormalTok{modelo \textless{}{-}}\StringTok{ \textquotesingle{} CA =\textasciitilde{} 0.88*x1 + 0.77*x2 + 0.73*x3}
\StringTok{            PHC =\textasciitilde{} 0.85*x4 + 0.80*x5 + 0.82*x6}
\StringTok{            VE =\textasciitilde{} 0.62*x7 + 0.74*x8 + 0.72*x9}

\StringTok{            VE \textasciitilde{} {-}0.01*CA + 0.41*PHC}
\StringTok{          \textquotesingle{}}
\end{Highlighting}
\end{Shaded}

\subsection{Escenarios de simulación para el tamaño de muestra y kurtosis}

\captionof{chunk}{Combinaciones de tamaño de muestra y kurtosis}\label{escenarios}

\begin{Shaded}
\begin{Highlighting}[]
\NormalTok{casos \textless{}{-}}\StringTok{ }\KeywordTok{expand.grid}\NormalTok{(}\DataTypeTok{kurtosis=}\KeywordTok{c}\NormalTok{(}\DecValTok{0}\NormalTok{, }\FloatTok{0.62}\NormalTok{, }\FloatTok{6.65}\NormalTok{, }\FloatTok{21.41}\NormalTok{, }\FloatTok{13.92}\NormalTok{), }\DataTypeTok{n=}\KeywordTok{c}\NormalTok{(}\DecValTok{50}\NormalTok{, }\DecValTok{100}\NormalTok{, }\DecValTok{200}\NormalTok{, }\DecValTok{400}\NormalTok{, }\DecValTok{800}\NormalTok{))}
\end{Highlighting}
\end{Shaded}

\subsection{Simulación de datos}

\captionof{chunk}{Generación de datos para cada escenario}\label{simulacion}

\begin{Shaded}
\begin{Highlighting}[]
\NormalTok{datos \textless{}{-}}\StringTok{ }\KeywordTok{lapply}\NormalTok{(}\DecValTok{1}\OperatorTok{:}\DecValTok{2000}\NormalTok{, }\ControlFlowTok{function}\NormalTok{(x)\{}
\NormalTok{  data \textless{}{-}}\StringTok{ }\KeywordTok{mapply}\NormalTok{(simulateData, }\DataTypeTok{sample.nobs=}\NormalTok{casos}\OperatorTok{$}\NormalTok{n, }\DataTypeTok{kurtosis=}\NormalTok{casos}\OperatorTok{$}\NormalTok{kurtosis, }
                 \DataTypeTok{MoreArgs =} \KeywordTok{list}\NormalTok{(}\DataTypeTok{model=}\NormalTok{modelo),}
                 \DataTypeTok{SIMPLIFY =} \OtherTok{FALSE}\NormalTok{)}
  
  \KeywordTok{names}\NormalTok{(data) \textless{}{-}}\StringTok{ }\KeywordTok{paste}\NormalTok{(}\StringTok{"n"}\NormalTok{, casos}\OperatorTok{$}\NormalTok{n, }\StringTok{"k"}\NormalTok{, casos}\OperatorTok{$}\NormalTok{kurtosis, }\DataTypeTok{sep=}\StringTok{""}\NormalTok{)}
\NormalTok{  data}
\NormalTok{\})}

\NormalTok{casos\_resultados \textless{}{-}}\StringTok{ }\KeywordTok{expand.grid}\NormalTok{(}\DataTypeTok{x=}\DecValTok{1}\OperatorTok{:}\KeywordTok{length}\NormalTok{(datos), }\DataTypeTok{y=}\KeywordTok{names}\NormalTok{(datos[[}\DecValTok{1}\NormalTok{]]))}

\NormalTok{nombres \textless{}{-}}\StringTok{ }\KeywordTok{unique}\NormalTok{(casos\_resultados}\OperatorTok{$}\NormalTok{y) }\OperatorTok{\%\textgreater{}\%}\StringTok{ }\NormalTok{paste}
\NormalTok{datos \textless{}{-}}\StringTok{ }\KeywordTok{lapply}\NormalTok{(nombres, }\ControlFlowTok{function}\NormalTok{(y)\{}
  \KeywordTok{lapply}\NormalTok{(}\DecValTok{1}\OperatorTok{:}\KeywordTok{length}\NormalTok{(datos), }\ControlFlowTok{function}\NormalTok{(x)\{}
\NormalTok{    datos[[x]][[y]]}
\NormalTok{  \})}
\NormalTok{\})}

\KeywordTok{names}\NormalTok{(datos) \textless{}{-}}\StringTok{ }\NormalTok{nombres}
\end{Highlighting}
\end{Shaded}

\subsection{Estimación de los SEM}

\captionof{chunk}{Definición del modelo y estimación para cada conjunto de datos}\label{modelos_sem}

\begin{Shaded}
\begin{Highlighting}[]
\NormalTok{modelo \textless{}{-}}\StringTok{ \textquotesingle{} CA =\textasciitilde{} x1 + x2 + x3}
\StringTok{            PHC =\textasciitilde{} x4 + x5 + x6}
\StringTok{            VE =\textasciitilde{} x7 + x8 + x9}
\StringTok{            VE \textasciitilde{} CA + PHC \textquotesingle{}}

\NormalTok{modelos \textless{}{-}}\StringTok{ }\KeywordTok{lapply}\NormalTok{(datos, }\ControlFlowTok{function}\NormalTok{(x)\{}
  \KeywordTok{lapply}\NormalTok{(x, }\ControlFlowTok{function}\NormalTok{(y)\{}
    \KeywordTok{sem}\NormalTok{(modelo, }\DataTypeTok{data=}\NormalTok{y)}
\NormalTok{  \})}
\NormalTok{\})}
\end{Highlighting}
\end{Shaded}

\section{REFERENCIAS}

\hypertarget{refs}{}
\begin{cslreferences}
\leavevmode\hypertarget{ref-andreassen}{}%
Andreassen, T. W., Lorentzen, B. G., \& Olsson, U. H. (2006). The impact
of non-normality and estimation methods in SEM on satisfaction research
in marketing. \emph{Quality and Quantity}, \emph{40}(1), 39-58.

\leavevmode\hypertarget{ref-Beran2010StructuralEM}{}%
Beran, T. N., \& Violato, C. (2010). Structural equation modeling in
medical research: a primer. \emph{BMC Research Notes}, \emph{3},
267-267. Recuperado de
\url{https://www.ncbi.nlm.nih.gov/pmc/articles/PMC2987867/\#}

\leavevmode\hypertarget{ref-chin}{}%
Chin, W. W., Peterson, R. A., \& Brown, S. P. (2008). Structural
Equation Modeling in Marketing: Some Practical Reminders. \emph{Journal
of Marketing Theory and Practice}, \emph{16}(4), 287-298.
\url{https://doi.org/10.2753/MTP1069-6679160402}

\leavevmode\hypertarget{ref-Fleishman1978}{}%
Fleishman, A. I. (1978). A method for simulating non-normal
distributions. \emph{Psychometrika}, \emph{43}(4), 521-532.
\url{https://doi.org/10.1007/BF02293811}

\leavevmode\hypertarget{ref-foss}{}%
Foss, T., Jöreskog, K. G., \& Olsson, U. H. (2011). Testing structural
equation models: The effect of kurtosis. \emph{Computational Statistics
\& Data Analysis}, \emph{55}(7), 2263-2275. Recuperado de
\url{https://EconPapers.repec.org/RePEc:eee:csdana:v:55:y:2011:i:7:p:2263-2275}

\leavevmode\hypertarget{ref-gao}{}%
Gao, S., Mokhtarian, P. L., \& Johnston, R. A. (2008). Nonnormality of
Data in Structural Equation Models. \emph{Transportation Research
Record}, \emph{2082}(1), 116-124. Recuperado de
\href{\%20https://doi.org/10.3141/2082-14}{https://doi.org/10.3141/2082-14}

\leavevmode\hypertarget{ref-golob}{}%
Golob, T. F. (2011). \emph{Structural Equation Modeling For Travel
Behavior Research} (University of California Transportation Center,
Working Papers N.º qt2pn5j58n). Recuperado de University of California
Transportation Center website:
\url{https://ideas.repec.org/p/cdl/uctcwp/qt2pn5j58n.html}

\leavevmode\hypertarget{ref-bagozzi}{}%
Horton, R. L. (1980). Book Reviews : CAUSAL MODELS IN MARKETING by
Richard P. Bagozzi (New York: Wiley, 1980. 303 pp., \$14.95).
\emph{Journal of the Academy of Marketing Science}, \emph{8}(3),
304-305. \url{https://doi.org/10.1177/009207038000800316}

\leavevmode\hypertarget{ref-Hu1999}{}%
Hu, L. T., \& Bentler, P. M. (1999). Cutoff criteria for fit indexes in
covariance structure analysis: Conventional criteria versus new
alternatives. \emph{Structural Equation Modeling}, \emph{6}(1), 1-55.
\url{https://doi.org/10.1080/10705519909540118}

\leavevmode\hypertarget{ref-kaplan}{}%
Kaplan, D. (2008). \emph{Structural Equation Modeling: Foundations and
Extensions}. Recuperado de
\url{https://books.google.co.cr/books?id=MdYgAQAAQBAJ}

\leavevmode\hypertarget{ref-ggpubr}{}%
Kassambara, A. (2020). \emph{ggpubr: 'ggplot2' Based Publication Ready
Plots}. Recuperado de \url{https://CRAN.R-project.org/package=ggpubr}

\leavevmode\hypertarget{ref-Kenny2015}{}%
Kenny, D. A. (2015). \emph{SEM: Fit (David A. Kenny)}. Recuperado de
\url{http://www.davidakenny.net/cm/fit.htm}

\leavevmode\hypertarget{ref-muthen}{}%
Muthen, B., \& Kaplan, D. (1992). A comparison of some methodologies for
the factor analysis of non-normal Likert variables: A note on the size
of the model. \emph{British Journal of Mathematical and Statistical
Psychology}, \emph{45}(1), 19-30. Recuperado de
\url{https://onlinelibrary.wiley.com/doi/abs/10.1111/j.2044-8317.1992.tb00975.x}

\leavevmode\hypertarget{ref-PerformanceAnalytics}{}%
Peterson, B. G., \& Carl, P. (2020). \emph{PerformanceAnalytics:
Econometric Tools for Performance and Risk Analysis}. Recuperado de
\url{https://CRAN.R-project.org/package=PerformanceAnalytics}

\leavevmode\hypertarget{ref-R}{}%
R Core Team. (2020). \emph{R: A Language and Environment for Statistical
Computing}. Recuperado de \url{https://www.R-project.org/}

\leavevmode\hypertarget{ref-lavaan}{}%
Rosseel, Y. (2012). lavaan: An R Package for Structural Equation
Modeling. \emph{Journal of Statistical Software}, \emph{48}(2), 1-36.
Recuperado de \url{http://www.jstatsoft.org/v48/i02/}

\leavevmode\hypertarget{ref-RStudio}{}%
RStudio Team. (2015). \emph{RStudio: Integrated Development Environment
for R}. Recuperado de \url{http://www.rstudio.com/}

\leavevmode\hypertarget{ref-sura}{}%
Sura-Fonseca, R. (2020). \emph{Modelos de ecuaciones estructurales:
consecuencias de la asimetría positiva en los indicadores endógenos
sobre las estimaciones puntuales de sus coeficientes y la bondad de
ajuste}. Recuperado de
\url{http://www.kerwa.ucr.ac.cr/handle/10669/80716}

\leavevmode\hypertarget{ref-tarka}{}%
Tarka, P. (2018a). An overview of structural equation modeling: its
beginnings, historical development, usefulness and controversies in the
social sciences. \emph{Quality \& Quantity: International Journal of
Methodology}, \emph{52}(1), 313-354.
\url{https://doi.org/10.1007/s11135-017-0469-8}

\leavevmode\hypertarget{ref-Tarka}{}%
Tarka, P. (2018b). An overview of structural equation modeling: its
beginnings, historical development, usefulness and controversies in the
social sciences. \emph{Quality \& Quantity: International Journal of
Methodology}, \emph{52}(1), 313-354. Recuperado de
\url{https://www.ncbi.nlm.nih.gov/pmc/articles/PMC5794813/}

\leavevmode\hypertarget{ref-Vale1983}{}%
Vale, C. D., \& Maurelli, V. A. (1983). Simulating multivariate
nonnormal distributions. \emph{Psychometrika}, \emph{48}(3), 465-471.
\url{https://doi.org/10.1007/BF02293687}

\leavevmode\hypertarget{ref-ggplot2}{}%
Wickham, H. (2016). \emph{ggplot2: Elegant Graphics for Data Analysis}.
Recuperado de \url{https://ggplot2.tidyverse.org}

\leavevmode\hypertarget{ref-dplyr}{}%
Wickham, H., François, R., Henry, L., \& Müller, K. (2020). \emph{dplyr:
A Grammar of Data Manipulation}. Recuperado de
\url{https://CRAN.R-project.org/package=dplyr}

\leavevmode\hypertarget{ref-tidyr}{}%
Wickham, H., \& Henry, L. (2020). \emph{tidyr: Tidy Messy Data}.
Recuperado de \url{https://CRAN.R-project.org/package=tidyr}

\leavevmode\hypertarget{ref-kableExtra}{}%
Zhu, H. (2019). \emph{kableExtra: Construct Complex Table with 'kable'
and Pipe Syntax}. Recuperado de
\url{https://CRAN.R-project.org/package=kableExtra}
\end{cslreferences}

\end{document}
