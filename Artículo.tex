% Options for packages loaded elsewhere
\PassOptionsToPackage{unicode}{hyperref}
\PassOptionsToPackage{hyphens}{url}
\PassOptionsToPackage{dvipsnames,svgnames*,x11names*}{xcolor}
%
\documentclass[
]{article}
\usepackage{lmodern}
\usepackage{amssymb,amsmath}
\usepackage{ifxetex,ifluatex}
\ifnum 0\ifxetex 1\fi\ifluatex 1\fi=0 % if pdftex
  \usepackage[T1]{fontenc}
  \usepackage[utf8]{inputenc}
  \usepackage{textcomp} % provide euro and other symbols
\else % if luatex or xetex
  \usepackage{unicode-math}
  \defaultfontfeatures{Scale=MatchLowercase}
  \defaultfontfeatures[\rmfamily]{Ligatures=TeX,Scale=1}
\fi
% Use upquote if available, for straight quotes in verbatim environments
\IfFileExists{upquote.sty}{\usepackage{upquote}}{}
\IfFileExists{microtype.sty}{% use microtype if available
  \usepackage[]{microtype}
  \UseMicrotypeSet[protrusion]{basicmath} % disable protrusion for tt fonts
}{}
\makeatletter
\@ifundefined{KOMAClassName}{% if non-KOMA class
  \IfFileExists{parskip.sty}{%
    \usepackage{parskip}
  }{% else
    \setlength{\parindent}{0pt}
    \setlength{\parskip}{6pt plus 2pt minus 1pt}}
}{% if KOMA class
  \KOMAoptions{parskip=half}}
\makeatother
\usepackage{xcolor}
\IfFileExists{xurl.sty}{\usepackage{xurl}}{} % add URL line breaks if available
\IfFileExists{bookmark.sty}{\usepackage{bookmark}}{\usepackage{hyperref}}
\hypersetup{
  colorlinks=true,
  linkcolor=blue,
  filecolor=Maroon,
  citecolor=Blue,
  urlcolor=blue,
  pdfcreator={LaTeX via pandoc}}
\urlstyle{same} % disable monospaced font for URLs
\usepackage[margin=1in]{geometry}
\usepackage{color}
\usepackage{fancyvrb}
\newcommand{\VerbBar}{|}
\newcommand{\VERB}{\Verb[commandchars=\\\{\}]}
\DefineVerbatimEnvironment{Highlighting}{Verbatim}{commandchars=\\\{\}}
% Add ',fontsize=\small' for more characters per line
\usepackage{framed}
\definecolor{shadecolor}{RGB}{248,248,248}
\newenvironment{Shaded}{\begin{snugshade}}{\end{snugshade}}
\newcommand{\AlertTok}[1]{\textcolor[rgb]{0.94,0.16,0.16}{#1}}
\newcommand{\AnnotationTok}[1]{\textcolor[rgb]{0.56,0.35,0.01}{\textbf{\textit{#1}}}}
\newcommand{\AttributeTok}[1]{\textcolor[rgb]{0.77,0.63,0.00}{#1}}
\newcommand{\BaseNTok}[1]{\textcolor[rgb]{0.00,0.00,0.81}{#1}}
\newcommand{\BuiltInTok}[1]{#1}
\newcommand{\CharTok}[1]{\textcolor[rgb]{0.31,0.60,0.02}{#1}}
\newcommand{\CommentTok}[1]{\textcolor[rgb]{0.56,0.35,0.01}{\textit{#1}}}
\newcommand{\CommentVarTok}[1]{\textcolor[rgb]{0.56,0.35,0.01}{\textbf{\textit{#1}}}}
\newcommand{\ConstantTok}[1]{\textcolor[rgb]{0.00,0.00,0.00}{#1}}
\newcommand{\ControlFlowTok}[1]{\textcolor[rgb]{0.13,0.29,0.53}{\textbf{#1}}}
\newcommand{\DataTypeTok}[1]{\textcolor[rgb]{0.13,0.29,0.53}{#1}}
\newcommand{\DecValTok}[1]{\textcolor[rgb]{0.00,0.00,0.81}{#1}}
\newcommand{\DocumentationTok}[1]{\textcolor[rgb]{0.56,0.35,0.01}{\textbf{\textit{#1}}}}
\newcommand{\ErrorTok}[1]{\textcolor[rgb]{0.64,0.00,0.00}{\textbf{#1}}}
\newcommand{\ExtensionTok}[1]{#1}
\newcommand{\FloatTok}[1]{\textcolor[rgb]{0.00,0.00,0.81}{#1}}
\newcommand{\FunctionTok}[1]{\textcolor[rgb]{0.00,0.00,0.00}{#1}}
\newcommand{\ImportTok}[1]{#1}
\newcommand{\InformationTok}[1]{\textcolor[rgb]{0.56,0.35,0.01}{\textbf{\textit{#1}}}}
\newcommand{\KeywordTok}[1]{\textcolor[rgb]{0.13,0.29,0.53}{\textbf{#1}}}
\newcommand{\NormalTok}[1]{#1}
\newcommand{\OperatorTok}[1]{\textcolor[rgb]{0.81,0.36,0.00}{\textbf{#1}}}
\newcommand{\OtherTok}[1]{\textcolor[rgb]{0.56,0.35,0.01}{#1}}
\newcommand{\PreprocessorTok}[1]{\textcolor[rgb]{0.56,0.35,0.01}{\textit{#1}}}
\newcommand{\RegionMarkerTok}[1]{#1}
\newcommand{\SpecialCharTok}[1]{\textcolor[rgb]{0.00,0.00,0.00}{#1}}
\newcommand{\SpecialStringTok}[1]{\textcolor[rgb]{0.31,0.60,0.02}{#1}}
\newcommand{\StringTok}[1]{\textcolor[rgb]{0.31,0.60,0.02}{#1}}
\newcommand{\VariableTok}[1]{\textcolor[rgb]{0.00,0.00,0.00}{#1}}
\newcommand{\VerbatimStringTok}[1]{\textcolor[rgb]{0.31,0.60,0.02}{#1}}
\newcommand{\WarningTok}[1]{\textcolor[rgb]{0.56,0.35,0.01}{\textbf{\textit{#1}}}}
\usepackage{graphicx}
\makeatletter
\def\maxwidth{\ifdim\Gin@nat@width>\linewidth\linewidth\else\Gin@nat@width\fi}
\def\maxheight{\ifdim\Gin@nat@height>\textheight\textheight\else\Gin@nat@height\fi}
\makeatother
% Scale images if necessary, so that they will not overflow the page
% margins by default, and it is still possible to overwrite the defaults
% using explicit options in \includegraphics[width, height, ...]{}
\setkeys{Gin}{width=\maxwidth,height=\maxheight,keepaspectratio}
% Set default figure placement to htbp
\makeatletter
\def\fps@figure{htbp}
\makeatother
\setlength{\emergencystretch}{3em} % prevent overfull lines
\providecommand{\tightlist}{%
  \setlength{\itemsep}{0pt}\setlength{\parskip}{0pt}}
\setcounter{secnumdepth}{-\maxdimen} % remove section numbering


%\usepackage{fancyhdr}
%\pagestyle{fancy}
%\rhead{\includegraphics[width = 1\textwidth]{marca.jpg}}


\usepackage{geometry}
\geometry{a4paper, left=10mm, right=10mm, bottom=15mm}
\usepackage{setspace}
\doublespacing
\usepackage[spanish]{babel}
\usepackage{color}
\usepackage{xcolor}
\usepackage{framed}
\colorlet{shadecolor}{gray!20}
\setcounter{secnumdepth}{0}
\usepackage{sectsty}


\chapternumberfont{\Large}
\chaptertitlefont{\Large}
\setcounter{tocdepth}{5}
\setcounter{secnumdepth}{5}
\setlength{\footskip}{20pt}%Esto sube el número de página
\usepackage{graphics}
\usepackage{setspace} %paquete para el doble espaciado
\doublespacing %inicia el doble espaciado
 %Esto quita el punto final en la numeracion de cada seccion
\usepackage{tocloft}

\usepackage{titlesec}
\titleformat{\section}
{\Large\bfseries}{\thesection}{0.5em}{}
\titleformat{\subsection}
{\large\bfseries}{\thesubsection}{0.5em}{}
\titleformat{\subsubsection}
{\normalsize\bfseries}{\thesubsubsection}{0.5em}{}
\titleformat{\paragraph}
{\normalsize\bfseries}{\theparagraph}{0.5em}{}
\renewcommand\cftsecaftersnum{}
\renewcommand\thesection{\arabic{section}}
\renewcommand\thesubsection{\thesection.\arabic{subsection}}
\usepackage{caption}
\usepackage{fancyhdr}
\pagestyle{fancy}
\fancyhf{}
\fancyhead[R]{\rightmark}
\fancyfoot[R]{\thepage}
%\fancyfoot[C]{Teléfono  2511-1400    /    posgrado@sep.ucr.ac.cr  /   www.sep.ucr.ac.cr}
\setlength{\headheight}{21.9pt}
\renewcommand\sectionmark[1]{%
\markright{\thesection\ #1}}
%\renewcommand{\footrulewidth}{0.4pt}


%\renewcommand{\footnoterule}{%
%  \kern -1pt
%  \hrule width \textwidth height 1pt
%  \kern 4pt
%}


%MARCA DE AGUA
%\usepackage{graphicx}
% \usepackage{fancyhdr}
%  \pagestyle{fancy}
%  \setlength\headheight{28pt}
%   \fancyhead[L]{\includegraphics[width=16cm]{marca.jpg}}
%   \fancyfoot[LE,RO]{}

\usepackage{booktabs}
\usepackage{longtable}
\usepackage{array}
\usepackage{multirow}
\usepackage{wrapfig}
%\usepackage{float}
\usepackage{colortbl}
\usepackage{pdflscape}
\usepackage{tabu}
\usepackage{threeparttable}
\usepackage{threeparttablex}
\usepackage[normalem]{ulem}
\usepackage{makecell}
\usepackage{xcolor}

\usepackage{tocloft}
\renewcommand{\cftsecleader}{\cftdotfill{\cftdotsep}}

%\renewcommand{\familydefault}{\sfdefault} %Para cambiar la fuente


%Para referenciar chunks
\usepackage{caption}
\usepackage{floatrow}
\floatsetup[figure]{capposition=top}
\floatsetup[table]{capposition=top}
\floatplacement{figure}{H}
\floatplacement{table}{H}

\DeclareNewFloatType{chunk}{placement=H, fileext=chk, name=}
\captionsetup{options=chunk}
\renewcommand{\thechunk}{Código~\arabic{chunk}}
\makeatletter
\@addtoreset{chunk}{section}
\makeatother
\usepackage{amssymb}
\usepackage{booktabs}
\usepackage{longtable}
\usepackage{array}
\usepackage{multirow}
\usepackage{wrapfig}
\usepackage{float}
\usepackage{colortbl}
\usepackage{pdflscape}
\usepackage{tabu}
\usepackage{threeparttable}
\usepackage{threeparttablex}
\usepackage[normalem]{ulem}
\usepackage{makecell}
\usepackage{xcolor}
\newlength{\cslhangindent}
\setlength{\cslhangindent}{1.5em}
\newenvironment{cslreferences}%
  {\setlength{\parindent}{0pt}%
  \everypar{\setlength{\hangindent}{\cslhangindent}}\ignorespaces}%
  {\par}

\title{UNIVERSIDAD DE COSTA RICA\\
SISTEMA DE ESTUDIOS DE POSGRADO\\
ESCUELA DE ESTADÍSTICA\\
~\\
~\\}
\usepackage{etoolbox}
\makeatletter
\providecommand{\subtitle}[1]{% add subtitle to \maketitle
  \apptocmd{\@title}{\par {\large #1 \par}}{}{}
}
\makeatother
\subtitle{EFECTOS DE LA KURTOSIS EN LA ESTIMACIÓN DE MODELOS DE
ECUACIONES ESTRUCTURALES BAJO DISTINTOS TAMAÑOS DE MUESTRA\\
~\\
~\\
~\\}
\author{\hfill\break
\hfill\break
\hfill\break
\hfill\break
\hfill\break
CÉSAR ANDRÉS GAMBOA SANABRIA B12672\\
ANDRÉS ESTEBAN ARGUEDAS LEIVA B40535\\
~\\
~\\
~\\
~\\
Ciudad Universitaria Rodrigo Facio, Costa Rica\\
~\\
~\\}
\date{2020}

\begin{document}
\maketitle

\pagenumbering{gobble}
\cleardoublepage

\newpage

\section*{RESUMEN}
\pagenumbering{gobble}

El Resumen

\textbf{\emph{Palabras clave}}:SEM, simulación, kurtosis, lavaan, R
\cleardoublepage

\section*{ABSTRACT}
\pagenumbering{gobble}

El resumen en inglés

\textbf{\emph{Palabras clave}}:SEM, simulation, kurtosis, lavaan, R
\cleardoublepage

\newpage

\pagenumbering{gobble}
\tableofcontents
\cleardoublepage
\pagenumbering{arabic}

\newpage

\section{INTRODUCCIÓN}

\subsection{Antecedentes}

Los Modelos de Ecuaciones Estructurales (en adelante SEM, por sus siglas
en inglés) representan un compendio de métodos estadísticos que buscan
estimar y examinar las relaciones existentes entre varias mediciones
fácilmente observables con conceptos más abstractos, denominados
constructos, que no pueden ser medidos ni analizados de manera directa.
Los SEM trabajan de una manera similar a los modelos de regresión más
clásicos, pero representan una mejora pues analizan las relaciones
causales lineales entre las variables involucradas al mismo tiempo que
los errores de medición (Beran \& Violato,
\protect\hyperlink{ref-Beran2010StructuralEM}{2010}).

Los SEM están presentes en multitud de campos de investigación. Según
Beran y Violato (2010), la cantidad de referencias a SEM en 1994 fueron
de 164, aumentaron a 343 en el 2000 y llegaron a 742 en el 2009, lo cual
es una señal de que muchos investigadores alrededor del mundo están
mostrando cada vez más interés en este tipo de estudios, pues
representan una potente herramienta para la investigación partiendo de
la teoría sustantiva que poseen los diversos estudios.

Uno de los principales campos de aplicación de los SEM son las ciencias
sociales, pues se busca explicar y/o predecir con un grado de validez el
comportamiento específico de una o varias personas en grupo. Teniendo
siempre en consideración (aunque de forma limitada) las condiciones que
afectan a cada individuo involucrado en el estudio, así como las
características propias de su entorno, los grupos de investigación
pueden definir factores y relaciones latentes que se encuentran
implícitas en el comportamiento humano. Este tipo de investigaciones
permite entender los fenómenos no solo de forma descriptiva, sino que es
posible también determinar relaciones de causalidad (Tarka,
\protect\hyperlink{ref-Tarka}{2018}).

Las variables indicadoras que sirven para construir las relaciones
implícitas en cuestión, llamadas comúnmente constructos, pueden llegar a
comportarse de manera muy diversa. Las ciencias sociales, al trabajar
con seres humanos, es común trabajar con variables cuyo comportamiento
es particularmente irregular, presentando valores muy distintos entre
los sujetos de estudio, generando de esta manera que los indicadores de
manera multivariada no sigan una distribución normal, lo cual representa
un supuesto fundamental al trabajar con SEM (Sura-Fonseca,
\protect\hyperlink{ref-sura}{2020}). El no cumplimiento de este supuesto
puede deberse, entre otras cosas, a medidas particularmente altas o
bajas de una medida estadística en específico: La kurtosis.

\subsection{El problema}

Si al trabajar con un SEM no se cumple el supuesto de normalidad
multivariada y además el modelo se estima vía máxima verosimilitud
podría cometerse el error de sobreestimar el estadístico chi-cuadrado,
el cual sirve de referencia para conocer la magnitud de la diferencia
entre la matriz de covariancias estimadas por el modelo con la obtenida
en la muestra. Lo anterior suele llevar al rechazar modelos que en
realidad resumen bien la realidad y además a la subestimación de los
errores asociados a los parámetros, lo cual genera interpretaciones
inadecuadas en lo referente a la significancia de las relaciones
planteadas por el modelo teórico. Considerar distintos niveles de
kurtosis permite conocer el impacto que esta medida tiene sobre las
estimaciones de un SEM dependiendo del tamaño de muestra utilizado
(Muthen \& Kaplan, \protect\hyperlink{ref-muthen}{1992}).

\subsection{Objetivos del estudio}

La presente investigación busca estudiar el efecto que tienen distintos
niveles de kurtosis en varios tamaños de muestra sobre las estimaciones
de un SEM. Para ello, tomando como base un estudio de la Universidad de
California (Gao, Mokhtarian, \& Johnston,
\protect\hyperlink{ref-gao}{2008}) se plantean los siguientes objetivos:

\subsubsection{Objetivo general}

Comparar mediante un estudio de simulación las estimaciones de modelos
de ecuaciones estructurales en presencia de variables observadas con
niveles de kurtosis de 0, 0.62, 6.65, 21.41 y 13.92 en tamaños de
muestra de 50, 100, 200, 400 y 800.

\subsubsection{Objetivos específicos}

\textbf{1)} Definir como modelo poblacional el obtenido por Sura-Fonseca
(2020) con dos variables exógenas y una endógena con tres variables
indicadoras cada uno \textbf{(página 99 de la tesis)} como modelo de
referencia teórico cuyas cargas factoriales se utilizarán para la
generación de los datos simulados.

\textbf{2)} Medir el posible sesgo causado en la estimación de los
modelos mediante el estadístico chi-cuadrado del modelo y la raíz del
cuadrado medio de error de aproximación (RMSEA), la raíz de residuos de
cuadrado medio estandarizado (SRMR) y el índice de bondad de ajuste
(GFI).

\textbf{3)} Comparar los valores poblacionales de las cargas factoriales
con los obtenidos en las simulaciones.

\textbf{4)} Publicar en una revista científica con revisión por pares el
manuscrito final, en forma de un artículo científico.

\subsection{Metodología de la investigación}

\textbf{AQUÍ VA UN RESUMEN DE LO QUE PONEMOS EN LA SECCIÓN DE
METODOLOGÍA}

\subsection{Organización del estudio}

\textbf{AQUÍ VA UNA DESCRIPCIÓN BREVE DE CADA ETAPA DEL TRABAJO}

\newpage

\section{METODOLOGÍA}

\subsection{Casos de simulación}

\subsection{Generación de datos con kurtosis}

Los datos fueron simulados mediante la función \texttt{simulateData()}
del paquete \texttt{lavaan} (Rosseel,
\protect\hyperlink{ref-lavaan}{2012}), el cual utiliza el método
propuesto por Vale y Maurelli (Vale \& Maurelli,
\protect\hyperlink{ref-Vale1983}{1983}) para la simulación de datos no
normales multivariados. Este método, comúnmente conocido como VM, se
basa en el método propuesto por Fleishman (Fleishman,
\protect\hyperlink{ref-Fleishman1978}{1978}), el cual, con base en una
variable aleatoria distribuida como una normal estándar, permite simular
una variable con un promedio, variancia, asimetría y kurtosis dada. El
método VM permite especificar, adicionalmente, correlaciones entre las
variables a estimar. Para utilizar el método de Fleishman, para generar
una cierta variable aleatoria \(Y\), se utiliza la siguiente ecuación:

\begin{equation} \label{eq:defY}
  Y = a + bX + cX^2 + d X^3
\end{equation}

donde \(X \sim \mathcal{N} (0,1)\). Es decir, se puede generar una
variable no normal \(Y\), con sus primeros cuatro momentos iguales a
valores especificados, con base en los valores \(a\), \(b\), \(c\) y
\(d\) de la ecuación \ref{eq:defY}, con base en una variable normal
estándar \(X\) hasta su tercer potencia. Luego, para poder obtener los
valores de \(a\), \(b\), \(c\) y \(d\), se necesitan resolver las
siguientes ecuaciones de forma simultánea:

\begin{align}
  b^2 + 6bd + 2c^2 + 15d^2 -1 & = 0 \\
  2c (b^2 + 24bd + 105d^2 + 2) - \gamma_1 & = 0 \\
  24 \left(bd + c^2 (1 + b^2 + 28bd) + d^2 (12 + 48bd + 141c^2 + 225d^2) \right) - \gamma_2 & = 0
\end{align}

donde \(\gamma_1\) es la asimetría deseada y \(\gamma_2\) es la kurtosis
deseada, además se define \(a = -c\). Con base en las constantes
calculadas \(a\), \(b\), \(c\) y \(d\), además de una variable normal
estándar, se puede simular variables no normales. Para poder generalizar
el método de Fleishman a variables aleatorias multivariantes, Vale y
Maurelli proponen una generalización. Esta se basa, para el caso
bivariado, en la generación de dos variables aleatorias independientes,
\(X_1, X_2 \sim \mathcal{N} (0,1)\), para la cuales se obtienen las
constantes \(a\), \(b\), \(c\) y \(d\), para cada una de dichas
variables, como se describe en el método de Fleishman, obteniendo así el
vector \(w^\prime_1 = (a_1, b_1, c_1, d_1)\), para el caso de \(X_1\), y
el vector \(w^\prime_2 = (a_2, b_2, c_2, d_2)\), para el caso de
\(X_2\). Además, se definen los vectores
\(x_1^\prime = (1, X_1, X_1^2, X_1^3)\) y
\(x_2^\prime = (1, X_2, X_2^2, X_2^3)\). Por lo tanto, se pueden crear
variables no normales, \(Y_1\) y \(Y_2\), como:

\begin{align*}
  Y_1 & = w_1^\prime x_1 \\
  Y_2 & = w_2^\prime x_2
\end{align*}

donde se puede verificar que:

\begin{align*}
  r_{Y_1, Y_2} = & \rho_{X_1, X_2} (b_1 b_2 + 3b_1 d_2 + 3d_1 b_2 + 9 d_1 d_2) \\
  & + \rho_{X_1, X_2}^2 (2 c_1 c_2) + \rho_{X_1, X_2}^3 (6 d_1 d_2)
\end{align*}

Y resolviendo esta ecuación en términos de \(\rho_{X_1, X_2}\) se puede
obtener una matriz de correlaciones para generar datos normales
multivariados, que pueden ser transformados en variables no normales
mediante el método de Fleishman.

\subsection{Modelo teórico a estimar}

\subsection{Medidas de bondad de ajuste}

\subsection{Simulación y estimación}

La simulación de los datos, junto con la estimación de los modelos, se
realizó mediante el paquete \texttt{lavaan} (Rosseel,
\protect\hyperlink{ref-lavaan}{2012}) usando el software R (R Core Team,
\protect\hyperlink{ref-R}{2020}) mediante la interfaz gráfica de RStudio
(RStudio Team, \protect\hyperlink{ref-RStudio}{2015}). Para e manejo de
bases de datos y demás visualizaciones fueron utilizados los paquetes
\texttt{ggplot2}(Wickham, \protect\hyperlink{ref-ggplot2}{2016}),
\texttt{tidyr} (Wickham \& Henry, \protect\hyperlink{ref-tidyr}{2020}),
\texttt{dplyr} (Wickham et~al., \protect\hyperlink{ref-dplyr}{2020}),
\texttt{ggpubr} (Kassambara, \protect\hyperlink{ref-ggpubr}{2020}),
\texttt{PerformanceAnalytics} (Peterson \& Carl,
\protect\hyperlink{ref-PerformanceAnalytics}{2020}) y
\texttt{kableExtra} (Zhu, \protect\hyperlink{ref-kableExtra}{2019}).

Para poder realizar la simulación deben seguirse varios pasos. Lo
primero es definir el modelo teórico poblacional que van a seguir los
datos simulados, como se describió en secciones anteriores este modelo
cuenta con dos variables exógenas y una endógena, cada una con tres
variables indicadoras. Los datos se generan mediante la función
\texttt{simulateData()} la cuál requiere especificar varios argumentos,
uno de ellos es el modelo poblacional, cuya sintaxis puede encontrarse
en el \ref{modelo}. Los otros dos argumentos a especificar son el tamaño
de muestra deseado y el nivel de kurtosis de interés, la definición de
estos escenarios se obtiene mediante el \ref{escenarios} y se muestran
en el cuadro \ref{tab:escenarios}:

\begin{table}[!h]

\caption{\label{tab:unnamed-chunk-6}\label{tab:escenarios}Escenarios de simulación}
\centering
\resizebox{\linewidth}{!}{
\fontsize{9}{11}\selectfont
\begin{tabu} to \linewidth {>{\raggedleft}X>{\raggedleft}X>{\raggedleft}X>{\raggedleft}X>{\raggedleft}X>{\raggedleft}X>{\raggedleft}X>{\raggedleft}X>{\raggedleft}X>{\raggedleft}X}
\toprule
kurtosis & n & kurtosis & n & kurtosis & n & kurtosis & n & kurtosis & n\\
\midrule
\rowcolor{gray!6}  0.00 & 50 & 0.00 & 100 & 0.00 & 200 & 0.00 & 400 & 0.00 & 800\\
0.62 & 50 & 0.62 & 100 & 0.62 & 200 & 0.62 & 400 & 0.62 & 800\\
\rowcolor{gray!6}  6.65 & 50 & 6.65 & 100 & 6.65 & 200 & 6.65 & 400 & 6.65 & 800\\
21.41 & 50 & 21.41 & 100 & 21.41 & 200 & 21.41 & 400 & 21.41 & 800\\
\rowcolor{gray!6}  13.92 & 50 & 13.92 & 100 & 13.92 & 200 & 13.92 & 400 & 13.92 & 800\\
\bottomrule
\multicolumn{10}{l}{\textit{Fuente:} Elaboración propia a partir del estudio de la Universidad de California (2008)}\\
\end{tabu}}
\end{table}

Con estos escenarios definidos, se generaron entonces, para cada
combinación de tamaño de muestra y kurtosis un total de 2000 conjuntos
de datos para cada uno. La sintaxis necesaria para generar esto se
muestra en el \ref{simulacion}. Una vez que se obtuvieron estos
conjuntos de datos, el siguiente paso es realizar la estimación de los
SEM con cada uno de ellos; para ello es necesario definir un modelo de
una forma similar a como se indicó en el \ref{modelo}, pero esta vez sin
los valores de las cargas factoriales, pues se busca conocer las
estimaciones a partir de los datos generados, este proceso se muestra en
el \ref{modelos_sem}.

\newpage

\section{RESULTADOS}

\subsection{Introducción}

\newpage

\section{CONCLUSIONES Y RECOMENDACIONES}

\subsection{Introducción}

\subsection{Conclusiones}

\subsection{Recomendaciones}

\newpage

\section{ANEXOS}

\subsection{Modelo poblacional para la función simulateData()}

\captionof{chunk}{Modelo poblacional para simular datos}\label{modelo}

\begin{Shaded}
\begin{Highlighting}[]
\NormalTok{modelo \textless{}{-}}\StringTok{ \textquotesingle{} CA =\textasciitilde{} 0.88*x1 + 0.77*x2 + 0.73*x3}
\StringTok{            PHC =\textasciitilde{} 0.85*x4 + 0.80*x5 + 0.82*x6}
\StringTok{            VE =\textasciitilde{} 0.62*x7 + 0.74*x8 + 0.72*x9}

\StringTok{            VE \textasciitilde{} {-}0.01*CA + 0.41*PHC}
\StringTok{          \textquotesingle{}}
\end{Highlighting}
\end{Shaded}

\subsection{Escenarios de simulación para el tamaño de muestra y kurtosis}

\captionof{chunk}{Combinaciones de tamaño de muestra y kurtosis}\label{escenarios}

\begin{Shaded}
\begin{Highlighting}[]
\NormalTok{casos \textless{}{-}}\StringTok{ }\KeywordTok{expand.grid}\NormalTok{(}\DataTypeTok{kurtosis=}\KeywordTok{c}\NormalTok{(}\DecValTok{0}\NormalTok{, }\FloatTok{0.62}\NormalTok{, }\FloatTok{6.65}\NormalTok{, }\FloatTok{21.41}\NormalTok{, }\FloatTok{13.92}\NormalTok{), }\DataTypeTok{n=}\KeywordTok{c}\NormalTok{(}\DecValTok{50}\NormalTok{, }\DecValTok{100}\NormalTok{, }\DecValTok{200}\NormalTok{, }\DecValTok{400}\NormalTok{, }\DecValTok{800}\NormalTok{))}
\end{Highlighting}
\end{Shaded}

\subsection{Simulación de datos}

\captionof{chunk}{Generación de datos para cada escenario}\label{simulacion}

\begin{Shaded}
\begin{Highlighting}[]
\NormalTok{datos \textless{}{-}}\StringTok{ }\KeywordTok{lapply}\NormalTok{(}\DecValTok{1}\OperatorTok{:}\DecValTok{2000}\NormalTok{, }\ControlFlowTok{function}\NormalTok{(x)\{}
\NormalTok{  data \textless{}{-}}\StringTok{ }\KeywordTok{mapply}\NormalTok{(simulateData, }\DataTypeTok{sample.nobs=}\NormalTok{casos}\OperatorTok{$}\NormalTok{n, }\DataTypeTok{kurtosis=}\NormalTok{casos}\OperatorTok{$}\NormalTok{kurtosis, }
                 \DataTypeTok{MoreArgs =} \KeywordTok{list}\NormalTok{(}\DataTypeTok{model=}\NormalTok{modelo),}
                 \DataTypeTok{SIMPLIFY =} \OtherTok{FALSE}\NormalTok{)}
  
  \KeywordTok{names}\NormalTok{(data) \textless{}{-}}\StringTok{ }\KeywordTok{paste}\NormalTok{(}\StringTok{"n"}\NormalTok{, casos}\OperatorTok{$}\NormalTok{n, }\StringTok{"k"}\NormalTok{, casos}\OperatorTok{$}\NormalTok{kurtosis, }\DataTypeTok{sep=}\StringTok{""}\NormalTok{)}
\NormalTok{  data}
\NormalTok{\})}

\NormalTok{casos\_resultados \textless{}{-}}\StringTok{ }\KeywordTok{expand.grid}\NormalTok{(}\DataTypeTok{x=}\DecValTok{1}\OperatorTok{:}\KeywordTok{length}\NormalTok{(datos), }\DataTypeTok{y=}\KeywordTok{names}\NormalTok{(datos[[}\DecValTok{1}\NormalTok{]]))}

\NormalTok{nombres \textless{}{-}}\StringTok{ }\KeywordTok{unique}\NormalTok{(casos\_resultados}\OperatorTok{$}\NormalTok{y) }\OperatorTok{\%\textgreater{}\%}\StringTok{ }\NormalTok{paste}
\NormalTok{datos \textless{}{-}}\StringTok{ }\KeywordTok{lapply}\NormalTok{(nombres, }\ControlFlowTok{function}\NormalTok{(y)\{}
  \KeywordTok{lapply}\NormalTok{(}\DecValTok{1}\OperatorTok{:}\KeywordTok{length}\NormalTok{(datos), }\ControlFlowTok{function}\NormalTok{(x)\{}
\NormalTok{    datos[[x]][[y]]}
\NormalTok{  \})}
\NormalTok{\})}

\KeywordTok{names}\NormalTok{(datos) \textless{}{-}}\StringTok{ }\NormalTok{nombres}
\end{Highlighting}
\end{Shaded}

\subsection{Estimación de los SEM}

\captionof{chunk}{Definición del modelo y estimación para cada conjunto de datos}\label{modelos_sem}

\begin{Shaded}
\begin{Highlighting}[]
\NormalTok{modelo \textless{}{-}}\StringTok{ \textquotesingle{} CA =\textasciitilde{} x1 + x2 + x3}
\StringTok{            PHC =\textasciitilde{} x4 + x5 + x6}
\StringTok{            VE =\textasciitilde{} x7 + x8 + x9}
\StringTok{            VE \textasciitilde{} CA + PHC \textquotesingle{}}

\NormalTok{modelos \textless{}{-}}\StringTok{ }\KeywordTok{lapply}\NormalTok{(datos, }\ControlFlowTok{function}\NormalTok{(x)\{}
  \KeywordTok{lapply}\NormalTok{(x, }\ControlFlowTok{function}\NormalTok{(y)\{}
    \KeywordTok{sem}\NormalTok{(modelo, }\DataTypeTok{data=}\NormalTok{y)}
\NormalTok{  \})}
\NormalTok{\})}
\end{Highlighting}
\end{Shaded}

\section{REFERENCIAS}

\hypertarget{refs}{}
\begin{cslreferences}
\leavevmode\hypertarget{ref-Beran2010StructuralEM}{}%
Beran, T. N., \& Violato, C. (2010). Structural equation modeling in
medical research: a primer. \emph{BMC Research Notes}, \emph{3},
267-267. Recuperado de
\url{https://www.ncbi.nlm.nih.gov/pmc/articles/PMC2987867/\#}

\leavevmode\hypertarget{ref-Fleishman1978}{}%
Fleishman, A. I. (1978). A method for simulating non-normal
distributions. \emph{Psychometrika}, \emph{43}(4), 521-532.
\url{https://doi.org/10.1007/BF02293811}

\leavevmode\hypertarget{ref-gao}{}%
Gao, S., Mokhtarian, P. L., \& Johnston, R. A. (2008). Nonnormality of
Data in Structural Equation Models. \emph{Transportation Research
Record}, \emph{2082}(1), 116-124. Recuperado de
\href{\%20https://doi.org/10.3141/2082-14}{https://doi.org/10.3141/2082-14}

\leavevmode\hypertarget{ref-ggpubr}{}%
Kassambara, A. (2020). \emph{ggpubr: 'ggplot2' Based Publication Ready
Plots}. Recuperado de \url{https://CRAN.R-project.org/package=ggpubr}

\leavevmode\hypertarget{ref-muthen}{}%
Muthen, B., \& Kaplan, D. (1992). A comparison of some methodologies for
the factor analysis of non-normal Likert variables: A note on the size
of the model. \emph{British Journal of Mathematical and Statistical
Psychology}, \emph{45}(1), 19-30. Recuperado de
\url{https://onlinelibrary.wiley.com/doi/abs/10.1111/j.2044-8317.1992.tb00975.x}

\leavevmode\hypertarget{ref-PerformanceAnalytics}{}%
Peterson, B. G., \& Carl, P. (2020). \emph{PerformanceAnalytics:
Econometric Tools for Performance and Risk Analysis}. Recuperado de
\url{https://CRAN.R-project.org/package=PerformanceAnalytics}

\leavevmode\hypertarget{ref-R}{}%
R Core Team. (2020). \emph{R: A Language and Environment for Statistical
Computing}. Recuperado de \url{https://www.R-project.org/}

\leavevmode\hypertarget{ref-lavaan}{}%
Rosseel, Y. (2012). lavaan: An R Package for Structural Equation
Modeling. \emph{Journal of Statistical Software}, \emph{48}(2), 1-36.
Recuperado de \url{http://www.jstatsoft.org/v48/i02/}

\leavevmode\hypertarget{ref-RStudio}{}%
RStudio Team. (2015). \emph{RStudio: Integrated Development Environment
for R}. Recuperado de \url{http://www.rstudio.com/}

\leavevmode\hypertarget{ref-sura}{}%
Sura-Fonseca, R. (2020). \emph{Modelos de ecuaciones estructurales:
consecuencias de la asimetría positiva en los indicadores endógenos
sobre las estimaciones puntuales de sus coeficientes y la bondad de
ajuste}. Recuperado de
\url{http://www.kerwa.ucr.ac.cr/handle/10669/80716}

\leavevmode\hypertarget{ref-Tarka}{}%
Tarka, P. (2018). An overview of structural equation modeling: its
beginnings, historical development, usefulness and controversies in the
social sciences. \emph{Quality \& Quantity: International Journal of
Methodology}, \emph{52}(1), 313-354. Recuperado de
\url{https://www.ncbi.nlm.nih.gov/pmc/articles/PMC5794813/}

\leavevmode\hypertarget{ref-Vale1983}{}%
Vale, C. D., \& Maurelli, V. A. (1983). Simulating multivariate
nonnormal distributions. \emph{Psychometrika}, \emph{48}(3), 465-471.
\url{https://doi.org/10.1007/BF02293687}

\leavevmode\hypertarget{ref-ggplot2}{}%
Wickham, H. (2016). \emph{ggplot2: Elegant Graphics for Data Analysis}.
Recuperado de \url{https://ggplot2.tidyverse.org}

\leavevmode\hypertarget{ref-dplyr}{}%
Wickham, H., François, R., Henry, L., \& Müller, K. (2020). \emph{dplyr:
A Grammar of Data Manipulation}. Recuperado de
\url{https://CRAN.R-project.org/package=dplyr}

\leavevmode\hypertarget{ref-tidyr}{}%
Wickham, H., \& Henry, L. (2020). \emph{tidyr: Tidy Messy Data}.
Recuperado de \url{https://CRAN.R-project.org/package=tidyr}

\leavevmode\hypertarget{ref-kableExtra}{}%
Zhu, H. (2019). \emph{kableExtra: Construct Complex Table with 'kable'
and Pipe Syntax}. Recuperado de
\url{https://CRAN.R-project.org/package=kableExtra}
\end{cslreferences}

\end{document}
